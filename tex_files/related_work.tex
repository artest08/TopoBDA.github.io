\section{Related Work}
This section summarizes the literature on three key aspects of autonomous driving and advanced driver-assistance systems (ADAS): \emph{Lane Divider Detection}, \emph{HDMap Element Prediction}, and the \emph{Road Topology Problem and Centerline Concept}. Lane Divider Detection focuses on accurately identifying lane boundaries to ensure safe and reliable vehicle navigation. HDMap Element Prediction, which also encompasses lane divider detection, involves forecasting the presence and attributes of high-definition map elements. These studies often utilize multi-camera setups that provide 360-degree coverage, which is essential for precise localization and path planning. Additionally, the Road Topology Problem and Centerline Concept, both integral to HDMap Element Prediction, address the challenges of understanding and representing the road network's structure. Together, these components form the backbone of contemporary research aimed at enhancing the safety and efficiency of autonomous vehicles.


\subsection{Lane Divider Detection} Lane divider detection methods can be categorized into two primary subcategories: perspective view methods and 3D lane divider methods.

\subsubsection{Perspective View Methods} Perspective view (PV) methods focus on detecting lane dividers from the PV and projecting them onto the ground using a homography matrix under the flat surface assumption. Despite the presence of various lane divider instances, semantic approaches are practical and effective, assuming a constant number of lane divider instances (e.g., 1\textsuperscript{st} left, 2\textsuperscript{nd} left, 1\textsuperscript{st} right, 2\textsuperscript{nd} right lane dividers). SCNN \cite{pan2018spatial} adheres to this methodology and introduces a module that sequentially processes the rows and columns of the feature map. UFLD \cite{qin2022ultra} transitions the formulation from pixel-based to grid-based, proposing row-based and column-based anchor formulations to enhance inference speed. LaneATT \cite{tabelini2021keep}, inspired by object detection studies, devises an anchor concept specifically for lanes. The emergence of the CurveLanes dataset \cite{xu2020curvelane} has led to the adoption of instance-segmentation-based methods for the PV domain. CondLaneNet \cite{liu2021condlanenet} introduces lane-specific methodologies such as offset prediction and row-wise formulation on top of the instance mask formulation in the PV domain. PolyLaneNet \cite{tabelini2021polylanenet} and BezierLaneNet \cite{feng2022rethinking} employ polynomial and Bezier curve representations, respectively, to reduce post-processing efforts and improve curve learning. 

\subsubsection{3D Lane Divider Methods} Recent studies have shifted towards directly predicting the 3D locations of lane dividers with the introduction of 3D lane divider datasets such as \cite{yan2022once}, OpenLane \cite{chen2022persformer}, and Apollo 3D Synthetic Lane \cite{guo2020gen}. This approach addresses the limitations of PV methods, specifically the flat world assumption due to the absence of depth information. Persformer \cite{chen2022persformer} utilizes a deformable attention-based decoder with Inverse Perspective Mapping (IPM) and formulates a 3D anchor concept. CurveFormer \cite{bai2023curveformer} employs a sparse query design with a deformable attention mechanism and predicts polynomial parameters in the BEV domain. BEV-LaneDet \cite{wang2023bev} uses a keypoint concept with predicted offsets and groups the keypoints of the same lane instance with an embedding concept. PETRV2 \cite{liu2023petrv2} extends its sparse query design for lane detection. M2-3DLaneNet explores the benefits of integrating lidar and camera sensors. Another study \cite{chen2023efficient} follows an instance-offset formulation in the BEV domain and aggregates offsets with a voting mechanism. LATR \cite{luo2023latr} introduces an end-to-end 3D lane detection framework that directly detects 3D lanes from front-view images using lane-aware queries and dynamic 3D ground positional embedding, significantly improving accuracy and efficiency over previous methods. GLane3D \cite{Ozturk2025GLane3D} utilizes a graph-based approach with keypoints and directed connections, achieving enhanced cross-dataset generalization performance.

\subsection{HDMap Element Prediction}

The prediction of High-Definition Map (HDMap) elements, such as lane dividers, road dividers, and pedestrian crossings, is essential for autonomous driving. Various methods have been developed to enhance the accuracy and efficiency of HDMap element prediction. HDMapNet \cite{li2022hdmapnet} transforms Bird’s Eye View (BEV) semantic segmentation into BEV instance segmentation through extensive post-processing, leveraging predicted instance embeddings and directional information. VectorMapNet \cite{liu2023vectormapnet} introduces an end-to-end vectorized HD map learning pipeline that predicts sparse polylines from sensor observations, explicitly modeling spatial relationships between map elements, thus eliminating the need for dense rasterized segmentation and heuristic post-processing. Contrary to the autoregressive structure of VectorMapNet, MapTR \cite{liao2022maptr} directly predicts points on polylines or polygons using a permutation-invariant Hungarian matcher, combining point query and instance query concepts, achieving real-time inference speeds and robust performance in complex driving scenes. InstaGraM \cite{shin2023instagram} redefines polyline detection as a graph problem, where keypoints are vertices and their connections are edges, leveraging graph neural networks to enhance accuracy and robustness. MGMap \cite{liu2024mgmap} uses instance masks to generate map element queries and refine features with mask outputs, significantly improving performance over baseline methods. MapVR \cite{zhang2024online} rasterizes MapTR outputs and applies instance segmentation loss to address keypoint-based method limitations, enhancing performance without extra computational cost during inference. ADMap \cite{hu2024admap} employs instance interactive attention and vector direction difference loss to reduce point sequence jitter, enhancing map accuracy and stability. BeMapNet \cite{qiao2023end} models map elements as multiple piecewise curves using Bezier curves, eliminating the need for post-processing and achieving superior performance on existing benchmarks. StreamMapNet \cite{yuan2024streammapnet} explores the temporal aspects of HDMap element prediction, using temporal information as propagated instance queries and warped BEV features in a recurrent manner. MapTracker \cite{chen2025maptracker} formulates mapping as a tracking task, maintaining multiple memory latents to ensure consistent reconstructions over time, significantly outperforming existing methods on consistency-aware metrics. Recent advancements include PriorMapNet \cite{wang2024priormapnet}, which enhances online vectorized HD map construction by incorporating priors, and HIMap \cite{zhou2024himap}, which integrates point-level and element-level information to improve prediction accuracy. Additionally, the Multi-Session High-Definition Map-Monitoring System \cite{wijaya2023multi} employs machine learning algorithms to track and update map elements across multiple sessions, ensuring HD maps remain accurate and up-to-date. The Ultra-fast Semantic Map Perception \cite{xu2024ultra} leverages both camera and LiDAR data to achieve real-time performance, featuring an orthogonal projection subspace for fast semantic segmentation and a Bayesian framework for enhanced global semantic fusion.

\subsection{Road Topology Problem and Centerline Concept}

Road topology refers to the interrelationships among lanes, as well as their connections to traffic lights and signs. However, using lane dividers for this problem is inefficient, as each lane requires two separate lane dividers. Consequently, the concept of centerlines has emerged as a more efficient and natural representation of lanes.

\subsubsection{Centerline Concept}

STSU \cite{can2021structured} introduced a novel approach for extracting a directed graph of the local road network in bird’s-eye-view (BEV) coordinates from a single onboard camera image, significantly improving traffic scene understanding. CenterLineDet \cite{xu2022centerlinedet} uses a transformer network to detect lane centerlines with vehicle-mounted sensors, effectively handling complex graph topologies such as lane intersections. LaneGAP \cite{liao2023lane} models lane graphs path-wise, preserving lane continuity and improving the accuracy of lane graph construction. MapTRV2 \cite{liao2024maptrv2} enhances centerline prediction efficiency by treating lane centerlines as paths and incorporating semantic-aware shape modeling. SMERF \cite{luo2023augmenting} integrates Standard Definition (SD) maps by tokenizing map elements using a transformer encoder, and employs these tokens in the cross-attention mechanism of a transformer decoder to improve lane detection and topology prediction. In contrast, TopoSD \cite{yang2024toposd} uses both map tokens and feature maps extracted from SD maps to enrich BEV features. SMART \cite{ye2025smart} uniquely leverages SD and satellite maps to learn robust map priors, enhancing lane topology reasoning for autonomous driving without relying on consistent sensor configurations. LaneSegNet \cite{li2023lanesegnet} introduces the concept of lane segments, combining geometry and topology information to provide a comprehensive representation of road structures.

\subsubsection{Road Topology}

TopoNet \cite{li2023graph} proposes a graph neural network architecture that models the relationships between centerlines and traffic elements. It incorporates prior relational knowledge to enhance feature interactions. TopoMLP \cite{wu2023topomlp} introduces an advanced pipeline for understanding driving topology by incorporating lane coordinates into the topology framework and using an L1 loss function to refine the interaction points. CGNet \cite{han2024continuity} focuses on preserving the continuity of centerline graphs and improving topology accuracy through modules like Junction Aware Query Enhancement and Bezier Space Connection. Topo2D \cite{li2024enhancing} integrates 2D lane priors to improve 3D lane detection and topology reasoning. TopoLogic \cite{fu2024topologic} proposes managing lane topology relationships by combining geometric lane distance with similarity-based topology relationships. TopoFormer \cite{lv2024t2sg} introduces a lane aggregation layer that leverages geometric distance in driving self-attention, along with a counterfactual intervention layer to improve reasoning by considering alternative scenarios and their causal impacts.